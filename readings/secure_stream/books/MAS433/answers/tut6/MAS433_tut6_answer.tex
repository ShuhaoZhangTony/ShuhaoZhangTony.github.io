\documentclass[11pt]{article}
\usepackage{amsmath,amsfonts,amsthm,amssymb}
\usepackage{exscale}
\usepackage{graphicx}

\setlength{\parindent}{0pt}
\setlength\topmargin{0in}
\setlength\headheight{0in}
\setlength\headsep{0.3in}
\setlength\textheight{8.5in}
\setlength\textwidth{6.3in}
\setlength\oddsidemargin{0in}
\setlength\evensidemargin{0in}
\setlength{\tabcolsep}{3pt}
\setlength\parskip{1ex plus 0.5ex minus 0.2ex}
\def\mod{\operatorname{mod}}




\title{MAS 433 Tutorial 6}
\author{Wang Xueou (087199E16)}
\begin{document}
\maketitle
\textbf{Question 1} Solution:\\
Addition group is not secure. In group $(Z_p,+)$, we have the following key exchange protocol:\\
\begin{center}
\begin{tabular}{ll|l}
 &Alice &Bob \\
step 1:&generate random number $r_a$ & generate random number $r_b$ \\
step 2:&compute $Y_a=r_a \times g \mod p$ & compute $Y_b=r_b \times g \mod p$ \\
step 3:&send $Y_a$ to Bob & send $Y_b$ to Alice \\
step 4:&compute $K_a=r_a \times Y_b \mod p$ & compute $K_b=r_b \times Y_a \mod p$\\
\end{tabular}
\end{center}
Then we have $K_a=K_b=r_a \times r_b \times g \mod p$. However, this is not secure. The attacker may get $Y_a$ and $Y_b$, then he just constructs a table $T_1$ for $r_a^i=i,i=1,2,\cdots,p$ and compute $K_a^i= r_a^i \times Y_b \mod p$. Then he tries $r_b^j=j,j=1,2,\cdots,p$ and compute $K_b^j=r_b^j \times Y_a \mod p$ until $K_b^j=K_a^i$ for some $i$ in $T_1$. \\

\textbf{Question 2 } Solution: \\
$$
n = 161 = 7 \times 23 
$$
$$
\varphi(n)=(7-1) \times (23-1) = 132
$$
Since $e \cdot d =1 \mod \varphi(n)$, we apply the extended Euclidean algorithm to $(e, \varphi(n))$:\\
$$
\begin{array}{lll} 
132 &=& 26 \times 5 + 2 \\
5 &=& 2 \times 2 +1 \\
1 &=& 5 - 2 \times 2 \\
  &=& 5 - 2 \times (132 - 26 \times 5) \\
  &=& -2 \times 132 + 53 \times 5 \\
\end{array}
$$
So $5^{-1} \mod 132 \equiv 53$, i.e., $d=53$.\\

To decrypt $3$, we compute: \\
$$
\begin{array}{lll}
m &=& c^d \mod n \\
 &=& 3^{53} \mod 161 \\
 &=& (3^{13})^4 \times 3 \mod 161 \\
 &=& 101^4 \times 3 \mod 161 \\
 &=& 144 \times 3 \mod 161 \\
 &=& 110
\end{array}
$$
The message is 110. \\

\textbf{Question 3 } Solution:\\
\textbf{3.1. } \\
CBC mode: padded with random bits or constant bits. \\

SHA-1: \begin{enumerate}
 \item pad bit `1' to the end of the message 
 \item pad some zeros 
 \item pad the message length (in bits) 
 \item After padding, the overall length should be multiple of the block size
\end{enumerate}

CMAC: pad by bit `1' followed by some zero bits \\

RSA: \begin{enumerate}
 \item $m||o||r$, where, m is the plaintext message of $(n-k_0-k_1)$ bits, o is some zeros of length $k_1$, and $r$ is a random and secret number of $k_0$ bits 
 \item $X=(m||o)\oplus G(r)$, where $G$ is a random function 
 \item $Y=H(X)\oplus r$, where $H$ is also a random function
 \item Then encrypt the padded message $X||Y$
\end{enumerate}
\textbf{3.2. } \\
It is not secure to use small modulus $n$. RSA is based on the difficulty in factorizing large numbers. A small integer $n$ is easy to be factorized so that an attacker can get $p$ and $q$ easily. \\

It is not secure to use small public key exponent $e$. There may be two situations:\\
-Attack 1 Example\\
Suppose $e=3$, then for small $m$ (say, $m < n^{1/3}$), we have $c=m^3 \mod n = m^3$. This means $m$ can be recovered from $c$ easily. \\
-Attack 2 Example\\
Suppose $e=3$ and $m$ is large. The same message $m$ is sent to 3 different receivers, then we have \\
\begin{eqnarray}
c_1 &=& m^3 \mod n_1 \\
c_2 &=& m^3 \mod n_2 \\
c_3 &=& m^3 \mod n_3 
\end{eqnarray}
Then we can find some $m^\prime$ such that $m^\prime$ satisfies (1), (2) and (3). From Chinese Remainder Theorem, we know $m^3 \equiv m^\prime \mod n_1n_2n_3$, i.e., $m^3$ can be known. Thus $m$ can be recovered easily. \\

It is not secure to use small private key $d$. The value of $d$ must be large for security reason. Otherwise the following attacks can be applied:\\
$\bullet$ Brute force attack: the size of $d$ should be more than 128 bits. The complexity is $2^{128}$ \\
$\bullet$ Advanced attack: \\
-If $d<n^{0.25}$, $d$ can be recovered from $e$ and $n$ easily. \\
-If $d<n^{0.292}$, $d$ can be recovered from $e$ and $n$ easily. \\
-It is conjected that if $d<n^{0.5}$, $d$ can be recovered from $e$ and $n$ easily. \\

\textbf{Question 4 } Solution: \\
\textbf{4.1. }\\
Let $g=gcd(p-1,q-1)$. Since $e\cdot d \equiv 1 \mod(\lambda(n))$, we can write $e\cdot d = \beta \lambda(n) + 1$ \\
Let $x=c^d \mod n$,then \\
$$
\begin{array}{lll}
x \mod p &=& ((m^e)^d \mod n)\mod p \\
         &=& (m^e)^d \mod p \\
         &=& m^{\frac{\beta (p-1)(q-1)}{g}+1} \mod p
\end{array}
$$
If $m$ and $p$ are co-prime, by Fermat's little theorem: $m^{p-1} \mod p = 1$, then
\begin{align} \label{eq:mpcoprime}
x \mod p &= m^{\frac{\beta (p-1)(q-1)}{g}+1} \mod p \nonumber \\
         &= (m^{p-1} \mod p)^{\frac{\beta (q-1)}{g}} m \mod p \nonumber \\
         &= m \mod p
\end{align}
If $m$ is a multiple of $p$, then 
\begin{eqnarray} \label{eq:mpncoprime}
x \mod p &=& m^{\frac{\beta (p-1)(q-1)}{g}+1} \mod p \nonumber \\
         &=& 0 \nonumber \\
         &=& m \mod p
\end{eqnarray}
From~\eqref{eq:mpcoprime} and~\eqref{eq:mpncoprime}, we have:
\begin{eqnarray} \label{eq:p}
x \equiv m \mod p
\end{eqnarray}
Similarly: 
\begin{eqnarray} \label{eq:q}
x \equiv m \mod q
\end{eqnarray}
From~\eqref{eq:p}and~\eqref{eq:q} and Chinese Remainder Theorem: 
$$
x = m \mod pq =m
$$
\textbf{4.2. } \\
$$
n = 161 = 7 \times 23 
$$
$$
\lambda(n)=\frac{(7-1) \times (23-1)}{gcd(7-1,23-1)} = \frac{6 \times 22}{2}=66
$$
Since $e \cdot d =1 \mod \lambda(n)$, we apply the extended Euclidean algorithm to $(e, \lambda(n))$:\\
$$
\begin{array}{lll} 
66 &=& 13 \times 5 + 1 \\
1 &=& 66 - 13 \times 5 \\
\end{array}
$$
So $5^{-1} \mod 66 \equiv -13\mod 66 = 53$, i.e., $d=53$.\\

To decrypt $3$, we compute: \\
$$
\begin{array}{lll}
m &=& c^d \mod n \\
 &=& 3^{53} \mod 161 \\
 &=& (3^{13})^4 \times 3 \mod 161 \\
 &=& 101^4 \times 3 \mod 161 \\
 &=& 144 \times 3 \mod 161 \\
 &=& 110
\end{array}
$$
The message is 110. \\

\textbf{Question 5 } Solution:\\
\textbf{5.1. } If they share the same modulus, each user can factorize $n$ easily from $e_i$ and $d_i$. Then each user can find the private keys of other users.\\
-Factorize $n$ from $e$ and $d$ \\
1) Since $e \cdot d \equiv 1 (\mod \varphi(n))$,
$$
e \cdot d -1 = \beta (p-1)(q-1),
$$
\; we know that $e \cdot d -1$ is even.\\
2) Randomly select an integer $x$, compute
$$
y = x^{(e \cdot d-1)/2} \mod n
$$
3) We know that $x^{e\cdot d-1} \mod n = x^{\beta \varphi(n)} \mod n = 1$ by Euler's Theorem.\\
4) from 2) and 3), we know that 
$$
y^2 =1 \mod n
$$
\; Thus $\gcd(y-1,n)$ gives a factor of $n$ if $y \ne \pm 1 (\mod n)$.\\

\textbf{5.2. } \\
Suppose 
$$
c_A = m^{e_A} \mod n
$$
$$
c_B = m^{e_B} \mod n
$$
Now we use the extended Euclidean Algorithm to find $a$ and $b$ such that $a \times e_A + b \times e_B = 1$. In fact, one of $a$ and $b$ must be negative, and suppose $a$ is negative (It is similar if $b$ is negative). We can get
\begin{align} \label{eq:commonmodulus}
c_A^a \times c_B^b &= (m^{e_A} \mod n)^a \times (m^{e_B} \mod n)^b \\
&= m^{a \times e_A + b \times e_B} \nonumber \\
&= m \nonumber 
\end{align}
In~\eqref{eq:commonmodulus}, we have $a$ is negative, so
$$
\begin{array}{lcl}
(m^{e_A} \mod n)^a &=& c_A^a \\
                   &=& (c_A^{-1})^{-a} \\
\end{array}
$$
where $c_A^{-1}$ is the inverse of $c_A \mod n$. \\

\textbf{Question 6 } Solution: \\
\textbf{6.1. }
$$
\begin{array}{ccc}
p - q &=& 2v \\
p^2 - pq &=& 2pv \\
p^2 - 2pv &=& n \\
p^2 - 2pv + v^2 &=& n + v^2 \\
(p  - v) ^2 &=& n + v^2 \\
p - v &=& \sqrt{n + v^2} (p \gg v)\\
p &=& v + \sqrt{n + v^2}
\end{array}
$$
Then we try all the possible values of $v$. If we guess $v$ correctly, $\sqrt{n + v^2}$ should be an integer. \\

\textbf{6.2. }
$p = v + \sqrt{n + v^2}$\\
Try $v = 1, 2, 3, \cdots, 9$, when $v = 9$,
$$
\begin{array} {lcl}
p &=& 9 + \sqrt{2189284635403183 + 9^2} \\
  &=& 46789801 \\
q &=& n/p \\
  &=& 46789783
\end{array}
$$ 

\textbf{Question 7 } Solution: \\
1. $m=\left\lfloor \sqrt{n} \right\rfloor = 506, Q(x)=(m+x)^2 - n$ \\
2. set $B=31$, the factor base is $\{-1,2,3,5,7,11,13,17,19,23,29,31\}$ \\
3. \\
\;\; 
$
\begin{array} {lcl}
x=-3 & \rightarrow & Q(x) = -3952 = (-1) \times 2^4 \times 13 \times 19 \\
x=-2 & \rightarrow & Q(x) = -2495 = (-1) \times 5 \times 19 \times 31 \\
x=-1 & \rightarrow & Q(x) = -1936 = (-1) \times 2^4 \times 11^2 \\
x=1  & \rightarrow & Q(x) = 88    = 2^3 \times 11 \\
x=5  & \rightarrow & Q(x) = 4160  = 2^6 \times 5 \times 13 \\
x=10 & \rightarrow & Q(x) = 9295 = 5 \times 11 \times 13^2 \\
x=13 & \rightarrow & Q(x) = 12400 = 2^4 \times 5^2 \times 31 \\
\end{array}
$ \\
4. Take $A=\{x=-3,-2,5,13\}$ \\
5. $y^2 = Q(-3) \times Q(-2) \times Q(5) \times Q(13) = (-1)^2 \times 2^{14} \times 5^4 \times 13^2 \times 19^2 \times 31^2 $ \\
\; $\Rightarrow y = (-1) \times 2^7 \times 5^2 \times 13 \times 19 \times 31 = -91105 \mod 256961$ \\
6. $z=(m-3) \times (m-2) \times (m+5) \times (m+13) = 75319 \mod 256961$\\
\; $\gcd(75319 + 91105, 256961) = 293$ \\
\; $\gcd(75319 - 91105, 256961) = 877$ \\
So $n=293 \times 877$\\


\textbf{Question 8 } Solution: \\
We have $c_1 = 10, c_2 = 159$. We use extended Euclidean Algorithm to find $c_1^{-1}$.
$$
\begin{array}{lcl}
227 &=& 22 \times 10 + 7 \\
10  &=& 1 \times 7 + 3 \\
7   &=& 2 \times 3 + 1 \\
1   &=& 7 - 2 \times 3 \\
    &=& 7 - 2 \times (10 - 1 \times 7) \\
    &=& -2 \times 10 + 3 \times 7 \\
    &=& -2 \times 10 + 3 \times (227 - 22 \times 10) \\
    &=& -68 \times 10 + 3 \times 227
\end{array}
$$
So $10^{-1} = -68 \mod 227 = 159$ \\
To dectypt $(c_1,c_2)$, we compute
$$
\begin{array}{lcl}
m &=& c_1^{-x} \cdot c_2 \mod p \\
  &=& 10^{-15} \cdot 159 \mod 227 \\
  &=& 159^15 \cdot 159 \mod 227 \\
  &=& 7
\end{array}
$$
The plaintext is 7.

\end{document}