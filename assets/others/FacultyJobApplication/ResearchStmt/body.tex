\paragraph{Multi-Query Optimization for Complex Event Processing in SAP ESP (ICDE'17)\\} 
As a PhD scholar in SAP Innovation Center Singapore from 2014 to 2018, I participated in improving SAP's stream processing platform, called SAP ESP.
The system aims at delivering real-time stream processing and analytics in time-critical applications. 
In SAP ESP, users can implement their complex event processing tasks, which continuously analysis real-time event streams and quickly identify pre-defined complex events.
%Being a sub-field of stream processing, 
%complex event processing (CEP) has been successfully applied in many areas such as Capital Markets,
%%~\cite{web}, 
%Internet of Things (IoT)
%%~\cite{fengjuan2013research} 
%and Data Center Intelligence.
%%~\cite{datacenter}. 
%Those domains are usually ``big data" applications with high velocity.
I have created MOTTO~\cite{motto,zhang2018multi}, a multi-query optimizer for complex event processing in SAP ESP as illustrated in Figure~\ref{figure:motto_optimizer}. 
MOTTO realizes more sharing opportunities by introducing pattern query decomposition and transformation. 
Those sharing techniques are also extented to support multiple nested pattern queries and pattern queries with different window constraints. 
Experiments demonstrate the efficiency of MOTTO with both real-world applications scenarios and sensitivity studies.

\begin{figure}[h]
\centering
\includegraphics[width=0.8\linewidth]{motto_optimizer}
\caption{Multi-query optimization workflow of MOTTO.}
\label{figure:motto_optimizer}
\end{figure}


\paragraph{Rvisiting the Design of Data Stream Processing Systems on Multi-Core Processors (ICDE'17)\\} 
For my Ph.D. dissertation, I was pioneering in discover the gaps between the design of modern stream processing systems and modern hardware architectures.
In particular, I summarize~\cite{profile} three common {design aspects} of modern DSPSs, including a) pipelined processing with message passing, b) on-demand data parallelism, and c) JVM-based implementation. 
Then, I conducted detailed profiling studies with micro benchmark on modern multi-socket multi-core by using Apache Strom and Flink as examples.
The results have shown that those designs have underutilized the scale-up architectures in these two key aspects:
a) The design of supporting both pipelined and data parallel processing {results} in a very complex massively parallel execution model in DSP systems, which causes high front-end stalls on a single CPU socket;
b) The design of continuous message passing mechanisms between operators severely limits the scalability of DSP systems on multi-socket multi-core architectures. 
For a concrete example, Figure~\ref{fig:I-trace} illustrates that the instruction footprint of both Storm and Flink exceed L1-Instruction cache, and hence leads frequent cache trashing.
Based on the profiling results, I have further proposed two optimizations~\cite{zhang2018efficient} and demonstrate promising performance improvements.


\begin{figure}
\centering
    \makebox[\textwidth][c]{
        \subfloat[Storm]{%
            \includegraphics*[width=0.4\textwidth]{storm-final.png}   %Replace by .pdf when submit
        }
        \subfloat[Flink]{%
            \includegraphics*[width=0.4\textwidth]{flink-final.png}  %Replace by .pdf when submit
        }
    }
    \caption{Instruction footprint between two consecutive invocations of the same function.}\label{fig:I-trace}

\end{figure}

\paragraph{\system: Scaling Data Stream Processing on Shared-Memory Multicore Architectures (SIGMOD'19)\\} 
My previous profiling study shows that existing DSPSs underutilized the underlying complex hardware microarchitecture and especially show poor scalability due to the unmanaged resource competition and unaware of NUMA effect. 
Hence, my subsequent effort spend on a complete revolution in designing next-generation stream
processing platform, namely BriskStream~\cite{briskstream}, specifically optimized for sharedmemory
multicore architectures.
To address NUMA effect, I have developed a new streaming execution plan optimization paradigm, namely Relative-Location Aware Scheduling (RLAS). 
%\system scales stream computation towards a hundred of cores under NUMA effect.
%The experiments on eight-sockets machines confirm that BriskStream significantly outperforms existing DSPSs up to an order of magnitude even without the tedious tuning process. 
Figure~\ref{figure:scale} shows the better scalability of
BriskStream than existing popular DSPSs on multi-socket servers by
taking Linear-Road Benchmark as an example. 
Unmanaged thread interference
and unnecessary remote memory access penalty prevent
existing DSPSs from scaling well on the modern multisockets machine.
The comprehensive experiments based on two eight-sockets machines confirm that \system significantly outperforms existing DSPSs up to an order of magnitude even without the tedious tuning process. 
In short, I showed how a DSPS, for the first time, scales stream computation towards a hundred of cores under NUMA effect.

\begin{figure}[h]
\centering
\includegraphics[width=0.5\linewidth]{scalability_lr}
\caption{System scalability comparison based on Linear-Road Benchmark.}
\label{figure:scale}
\end{figure}

%The discussion on the different aspects discussed above illustrate the challenges we may confront in building a high performance DSPSs that can effectively utilize modern multicore architectures.
%These aspects are tightly coupled with each other, and the redesign of a single component can directly affect the effectiveness of others. 
%Witnessing these problems, in this thesis, we study the problem of building scalable multicore DSPSs from a systematic perspective. 
%In particular, we discuss the design and implementation of two core components of DSPSs, including execution plan optimization and state management. 
%Throughout this thesis, we conduct comprehensive performance study and propose novel mechanisms to address the issues identified above. In addition, we also point out some future works in designing and implementing next-generation multicore DSPSs.

\paragraph{Towards Concurrent Stateful Stream Processing on Multicore Processors (ICDE'20)\\}
DSPS with transactional state management relieves users from managing state consistency by themselves,  
and has recently received attention from both academia and industry community. 
However, scaling stream processing while providing transactional state management 
on modern multicore processors is challenging. 
On the one hand, to achieve both low latency and high throughput, 
DSPSs can process multiple input events at the same time in order to aggressively exploit parallelism.
On the other hand, processing different events concurrently may lead to conflict accesses (reads and writes) to the same application state, hence leading to higher chances of violating transactional state consistency. 
To make things worse, more than simply guaranteeing the ACID properties preserved in the relational database systems, DSPSs further need to enforce the state access \emph{order} according to the input event {sequence}.
Witnessing those issues, I have developed \tsystem~\cite{tstream}, a new DSPS that can support highly scalable stream processing with transactional state consistency guarantee on multicores.
In order to take advantage of multicore architectures, \tsystem detaches the state management from the streaming computation logic, and performs its internal state maintenance asynchronously. 
By eliminating the expensive synchronization primitives, \tsystem aggressively extracts parallelism opportunities by revealing the operation dependencies at runtime.
%We evaluate \tsystem in detail on a modern 40-core machine.
The initial results show that \tsystem achieves several times higher throughput on average over existing solutions with similar or even smaller end-to-end processing latency.

\paragraph{Research Agenda\\}
The trend of more radical performance demand, complex analytics, and intensive state access have accelerated the development of next-generation data stream processing systems (DSPSs)~\cite{profile,briskstream,tstream}. 
Beyond what I have previously described, there are a number of emerging issues in data stream processing that we shall pay more attention to.

The Internet of Things (IoT) presents a novel computing architecture for data management. 
In such context, the ability to scale both out and up is crucial to effectively improve performance by orders of magnitude. 
How to use the more and more heterogeneous devices in the present of millions distributed entities remains unclear today.
My short term plan (2$\sim$3 years)  is hence to collaborate with my current postdoc advisor, Prof. Markl Volker (TU Berlin) to investigate the future stream processing system~\cite{nebulastream} designed for the IoT environment. 
Specifically, I plan to explore questions such as:

\begin{myenumerate}
\item \textbf{Privacy and security.}
In an IoT environment, data processing at the edge is highly exposed to security threats, e.g., a sensor may be hacked. 
Intel SGX is one of the popular Trusted Execution Environment (TEE) implementations, which may help to provide security guarantees to edge processing. 
The first step is to investigate the designing of important stream operations (e.g., filter, projection, windowing) running based on Intel SGX. 

\emph{The potential long term goal of this direction is a redesign towards more secure data processing platforms in the context of IoT environment~\cite{streamboxtz}.}
%To this end, we will implement and evaluate few stream operations (e.g., filter, projection, windowing) running based on Intel SGX.

\item \textbf{Transactional state management.}
Recent DSPSs can achieve excellent performance when processing large volumes of data under tight latency constraints. However, they sacrifice support for concurrent state access that eases the burden of developing stateful stream applications. One of the promising ways of managing concurrent state access during stream processing is to model state accesses as transactions. 
My current result~\cite{tstream} successfully achieves orders of magnitude improvement over the state of the art on shared-memory multicore processors. However, many questions are still remain unsolved, such as how to achieve the same performance improvement in the context of distributed environment. 
Another potential weakness is its persistency from crash failures. 
In contrast to traditional database, stream processing often needs to preserve event process ordering. 
How to efficiently provide high concurrency and fast durability to DSPSs while preserve event process ordering remains a non-trivial question.

\emph{The potential long term goal of this direction is a redesign towards a more reliable and deterministic transactional data stream processing system.}

\item \textbf{Machine learning and Data Mining.}
%Needs and opportunities for machine learning over fast data
%streams are stimulated by a rapidly growing
%number of industrial, transactional, sensor and
%other applications. 
%Beyond online interference,
The principal task of online machine learning is to learn a concept incrementally by processing labeled training examples one at a time. 
%After
%each data instance, we can update the model,
%after which the instance is discarded. 
Although the massively parallel processors of modern hardware provides the opportunities in providing very fast training process, there is little work on exploring how to accelerate future DSPSs in supporting real-time machine learning activities. 
Fast data exploration has also received many attentions recently~\cite{8919445}. 
It aims to present users preliminary results as fast as possible in order to improve user interactivity. 
However, little works or systems are able to achieve pleasant results, there are still huge potential to explore. For example, how to answer applications with user-defined functions quick remains an open question.

\emph{The potential long term goal of this direction is a redesign towards a more functional and convenient data stream processing system that supports online machine learning and fast data exploration.}

%We hence foresee its potential in
%accelerating future DSPSs in supporting continuous
%machine learning.

%\item 
%Scaling stream processing on modern hardware has received many attentions recently.
%A potential weakness, however, is its persistency from crash failures. 
%In contrast to traditional database, stream processing often needs to preserve event process ordering. 
%How to efficiently provide high concurrency and fast durability to DSPS while preserve event process ordering seems a non-trivial question. 
%\item 
%There is increasing interest in using multicore processors to accelerate stream processing.
%In particular, there are many past works on using multicore, GPGPU, FPGA to accelerate streaming window joins. 
%Streaming window aggregation is also computationally expensive, but receives insufficient attention in accelerating its performance by utilizing modern hardware. 
\end{myenumerate}

