%%Application demands
In the investment community, the \emph{time value of money}
%~\cite{value} 
states that money is always more valuable today than in the future. 
A similar concept of \emph{time value of data} is getting widely recognized in the data processing community -- insights are derived from processing data, and the value of insights diminishes very fast with time. 
Due to the increased automation in many domains such as telecommunications, health care, transportation, and retail, numerous data-intensive applications are deployed in real-world use cases. 
Those applications generally involve continuously {low-latency}, {complex analytics} over {massive data streams} and are often named as \emph{streaming applications}. 
Data stream processing system (DSPS) is a software that allows users to efficiently run their streaming applications in a scalable way.


%With the proliferation of high-rate data sources, numerous data-intensive applications are deployed in real-world use cases exhibiting requirements that cannot be satisfied by traditional batch processing models. 
%Some representative applications include real-time credit fraud detection~\cite{nguyen2005sense,phua2010comprehensive}, stock analysis~\cite{abadi2005design}, and edge device tracking~\cite{tonjes2014real}. 
%Those applications generally involve continuously {low-latency}, {complex analytics} over {massive data streams} and are often named as \emph{streaming applications}. 
%They have become ubiquitous due to the increased automation in many domains such as telecommunications, health care, transportation, retail.
%%System status
%With the proliferation of high-rate streaming sources, numerous streaming applications are deployed in real-world use cases~\cite{Transactions2018}. 
%Unlike conventional database management systems (DBMSs) that provide ACID guarantees for relational data storage, retrieval, and mining, 
%Modern DSPSs are featured in supporting continuous lower-latency analytics over real-time data streams. 
%For example, some popular open-sourced DSPSs
%~\cite{Storm,flink} 
%are able to achieve very low processing latency in the scale of milliseconds.
%Due to its unique characteristics, a large body of system research has focused on designing and implementing new DSPSs to meet the fast increasing and more diverging application demands. 
%Arguably starting from 2000's, DSPSs have been investigated by a large number of research groups in the database community~\cite{Aurora,TelegraphCQ,Borealis,s4,Storm,heron,flink,SparkStreaming,Samza,s4}, and leading enterprises including SAP~\cite{motto}, IBM~\cite{system-s}, Google~\cite{millwheel} and Microsoft~\cite{trill}.
